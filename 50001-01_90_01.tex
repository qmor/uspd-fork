% !TeX spellcheck = ru_RU
% !TeX encoding = UTF-8
\documentclass[a4paper,12pt]{article}
\usepackage[T2A]{fontenc}
\usepackage[utf8]{inputenc}
\usepackage{lastpage}
\usepackage[russian]{babel}
\usepackage{uspd}
\usepackage{ifthen}



%variables
\newboolean{needtableofcontents}
\setboolean{needtableofcontents}{true}
\newboolean{needmilitary}
\setboolean{needmilitary}{true}
\newcommand{\militarydep}{0101 ВП МО РФ}
\newcommand{\militarydephead}{Ф.Рыкин}
\newcommand{\docdecimalnumber}{ЕВИЛ.50001-01}
\newcommand{\preparedby}{Д.Зыкин}
\newcommand{\checkedby}{О.Мыкин}
\newcommand{\developdep}{931}
\newcommand{\developdephead}{В.Лютый}
\newcommand{\programnameshort}{СПО ОВА}
\newcommand{\programnamefull}{Специальное программное обеспечение обмена с внешними абонентами}
\newcommand{\approvedby}{Заместитель генерального директора}
\newcommand{\approvedbyname}{Б.Зеленый}
%end of variables



\begin{document}

\sloppy
\title{\programnamefull\\(\programnameshort)
~\\
Протоколы информационного обмена}

\uspdabstract{Настоящий документ определяет протоколы информационного обмена (входные и выходные данные) \programnameshort.}


\begin{uspd}{\docdecimalnumber~90~01}
	
\section{Общие положения}
Входные и выходные данные для \programnameshort представлены в файлов, содержащих JavaScript Object Notation (JSON) объекты. Все файлы, используемые при информационном обмене должны иметь кодировку UTF-8. Подробное описание указанных файлов приводится в соответствующих разделах настоящего документа.
\section{Входные данные}
\subsection{Пользовательские данные}
Пользовательские данные передаются в виде JPEG картинок
\section{Выходные данные}
\subsection{Пользовательские данные}
Пользовательские данные также передаются в виде JPEG картинок


\newpage
\section*{Перечень сокращений}
\begin{tabular}{p{17mm}cp{180mm}}	
	ПК  & & - персональный компьютер \\
	СПО & & - специальное программное обеспечение\\
\end{tabular}
\end{uspd}

\end{document}
